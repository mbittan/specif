\documentclass[a4paper, 11pt]{article}
 
\usepackage[utf8]{inputenc}
\usepackage{graphicx}
\usepackage[frenchb]{babel}
\usepackage{tikz}
\usetikzlibrary{arrows,automata}
\usepackage{makecell}

\begin{document}
 
\title{SPECIF\\Compte-rendu TP2}
\author{Maxime Bittan}
\date{02/03/2015}
 
\maketitle
 
\paragraph{Question 1}
Le code correspondant au métronome se trouve dans le fichier $metronome.lus$. Le code C généré présente un seul état, avec une seule transition vers celui-ci.

\paragraph{Question 2}
Les variables présentes dans le code C généré correspondent aux variable Lustre suivantes : 
\begin{itemize}
  \item \_V1 : reset
  \item \_V2 : delay
  \item \_V3 : true$\rightarrow$false
  \item \_V4 : init
  \item \_V5 : n
  \item \_V6 : hz
  \item \_V7 : state
  \item \_V8 : tic
  \item \_V9 : tac
\end{itemize}

\paragraph{}
Soit $D = \{first, init, hz, n, state, tic, tac\}$, avec $first$ initialisé à \texttt{true}.

\begin{tikzpicture}[->,>=stealth',shorten >=1pt,auto,node distance=2.8cm,
                    semithick]
  \tikzstyle{every state}=[fill=white,text=black]

  \node[initial,state] (s0) {$s_0$};
  \path (s0) edge [loop right] node 
        {\makecell[l]{[true]\\
            lire $reset, delay$ \\
            init $\leftarrow$ \textbf{if} first \textbf{then} reset \textbf{else} init $||$ reset;\\
            hz $\leftarrow$ \textbf{if} reset \textbf{then} delay \textbf{else} (\textbf{if} first \textbf{then} 0 \textbf{else} hz);\\
            n $\leftarrow$ \textbf{if} reset \textbf{then} delay \textbf{else} 
            (\textbf{if} first \textbf{then} hz-1 \textbf{else} \\
            (\textbf{if} init and n$>$0 \textbf{then} n-1 \textbf{else} hz-1));\\
            state $\leftarrow$ \textbf{if} n=0 \textbf{then}\\
            (\textbf{if} first \textbf{then} true \textbf{else} not state)\\
            \textbf{else}(\textbf{if} first \textbf{then} false \textbf{else} state);\\
            tic $\leftarrow$ \textbf{if} init and n=0 and hz$>$0 \textbf{then} state \textbf{else} false;\\
            tac $\leftarrow$ \textbf{if} init and n=0 and hz$>$0 \textbf{then} not state \textbf{else} false;\\
            first $\leftarrow$ false;}} (s0);
\end{tikzpicture}

\paragraph{Question 3}
Les valeurs lustres correspondant aux variables du code C généré par la commande lus2oc -2 sont les suivantes :
\begin{itemize}
  \item \_V1 : reset
  \item \_V2 : delay
  \item \_V3 : hz $>$ 0
  \item \_V4 : hz
  \item \_V5 : n$==$0
  \item \_V6 : n
  \item \_V7 : n $>$ 0
  \item \_V8 : tic
  \item \_V9 : tac
\end{itemize}
 
\end{document}
